% A sub−file (e.g. picture) using the ’standalone’ class:
% Use ’standalone’ as document class:

\documentclass{standalone}
% Load packages needed for this TeX file:
\usepackage{tikz}
% Surround TeX code with ’document’ environment as usually:
\usetikzlibrary{arrows, arrows.meta, calc, backgrounds, fit, positioning, shapes}
% Surround TeX code with ’document’ environment as usually:


%%%%%%%%%%%%%%%%%%%%%%%%%%%%%%%%%%%%%%%%%%%%%%%%%%%%%%%%%%%%%%%%%%%%%%%%%%%%%%%%
% Definitions
%%%%%%%%%%%%%%%%%%%%%%%%%%%%%%%%%%%%%%%%%%%%%%%%%%%%%%%%%%%%%%%%%%%%%%%%%%%%%%%%
\def\neuron_radius{0.75 cm}
\def\interneuron_radius{0.5 cm}

\definecolor{flexor}{RGB}{240,0,0}
\definecolor{extensor}{RGB}{0, 0,186}
\definecolor{inhibitory}{RGB}{152,62,106}
\definecolor{excitatory}{RGB}{0,150,93}

\tikzset{
% Color shades
my-node/.style = {
  circle, minimum size=\neuron_radius, inner sep=0.01cm, outer sep=0.01cm, draw=black, thick,
  double, font={\footnotesize}, text=white
},
no shape/.style = {rectangle, inner sep=0pt, draw=none, fill=none}
}


%%%%%%%%%%%%%%%%%%%%%%%%%%%%%%%%%%%%%%%%%%%%%%%%%%%%%%%%%%%%%%%%%%%%%%%%%%%%%%%%
% General Styles
%%%%%%%%%%%%%%%%%%%%%%%%%%%%%%%%%%%%%%%%%%%%%%%%%%%%%%%%%%%%%%%%%%%%%%%%%%%%%%%%
\tikzset{
  % Color shades
  flexor-shade/.style={
    fill=flexor!75,
  },
  extensor-shade/.style={
    fill=extensor!75,
  },
  excitatory-shade/.style={
    fill=excitatory!75,
  },
  inhibitory-shade/.style={
    fill=inhibitory!75,
  },
  interneuron-shade/.style={
    fill=inhibitory!75,
  },
  % Node styles
  neuron/.style n args={1}{
    #1,
    my-node,
  },
  neuron/.default={flexor},
  flexor/.style={
    my-node,
    flexor-shade,
  },
  extensor/.style={
    my-node,
    extensor-shade,
  },
  interneuron/.style={
    my-node,
    minimum size=\interneuron_radius,
    interneuron-shade,
  },
  excitatory/.style={
    my-node,
    excitatory-shade,
  },
  inhibitory/.style={
    my-node,
    inhibitory-shade,
  },
  % Edge styles
  neuron-edge/.style n args={2}{
    color=#1,
    thick,
    #2,
    on background layer,
  },
  neuron-edge/.default={green}{-{Latex[scale=1.0]}},
  flexor-edge/.style={
    neuron-edge={flexor}{-{Latex[scale=1.0]}},
  },
  extensor-edge/.style={
    neuron-edge={extensor}{-{Latex[scale=1.0]}},
  },
  inhibitory-edge/.style={
    neuron-edge={inhibitory}{-*},
  },
  excitatory-edge/.style={
    neuron-edge={excitatory}{-{Latex[scale=1.0]}},
  },
}

\begin{document}
% Add your TeX code, e.g. a picture:
\begin{tikzpicture}
  {{network}}
  % Add grid lines
  
  \begin{scope}[on background layer]
    \draw[color=black, thick, dashed, on background layer] (current bounding box.north) -- (current bounding box.south);
    \draw[color=black, thick, dashed, on background layer] (current bounding box.east) -- (current bounding box.west);
  \end{scope}
 
 % add legend
   \begin{scope}
     \node[matrix, below, draw, row sep=0.1 cm, column sep=0.5cm, text=black, minimum size=\interneuron_radius]
     (centers-legend)  at (current bounding box.south) {
         \node [flexor, label={[label distance=0.2cm]0:flexor center}, text=white] {F};
         & \node [excitatory, label={[label distance=0.2cm]0:excitatory population},
         text=white] {};
         & \node [no shape, label={[label distance=0.3cm]0:excitatory connections}] (glu) {};
         \draw[extensor-edge] (glu.south west) -- +(0.6, 0.0) {};
         \draw[flexor-edge] (glu.west) -- ++(0.6, 0.0);
         \draw[excitatory-edge] (glu.north west) -- ++(0.6, 0.0);
         \\
         \node [extensor, label={[label distance=0.2cm]0:extensor center}, text=white] {E};
         & \node [inhibitory, label={[label distance=0.2cm]0:inhibitory population},
         text=white] {};
         & \node [no shape, label={[label distance=0.3cm]0:inhibitory connections}] (gaba){};
         \draw[inhibitory-edge] (gaba.west) -- ++(0.6, 0.0);\\
       };
 \end{scope}
\end{tikzpicture}
\end{document}


%%% Local Variables:
%%% mode: latex
%%% TeX-master: t
%%% End:
